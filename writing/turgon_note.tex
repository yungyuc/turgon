\documentclass[11pt,dvips]{article}
%\documentclass[12pt,dvips]{article}
%\documentclass[preprint,dvips,numbers,sort&compress]{elsarticle}
%\documentclass[12pt,review,dvips,numbers,sort&compress]{elsarticle}
% Geometry.
\usepackage{geometry}
\geometry{a4paper,
left=2cm,
right=2cm,
top=2cm,
bottom=2cm,
}
% Global functionalities.
\usepackage[numbers,sort&compress]{natbib}
\usepackage{hyperref}
% encoding.
\usepackage[utf8]{inputenc}
% Mathematics.
\usepackage{amsmath}
\usepackage{amssymb}
\usepackage{amsfonts}
\usepackage{amsthm}
\usepackage{arydshln}
% Authoring.
%\usepackage{authblk}
\usepackage{graphicx}
\usepackage{subfigure}
\usepackage{color}
\usepackage{paralist}
\usepackage{multirow}
\usepackage{booktabs}
\usepackage{setspace}
%\usepackage{lmodern}

%\doublespacing

\graphicspath{{turgon_eps/}}

\renewcommand{\figurename}{Fig.}
\newcommand{\topcaption}{%
\setlength{\abovecaptionskip}{0pt}%
\setlength{\belowcaptionskip}{10pt}%
\caption}
\numberwithin{equation}{section}

\newcommand{\defeq}{\ensuremath{\buildrel {\text{def}}\over{=}}}

\title{
%
The Space-Time Conservation Element and Solution Method Schemes
%
}

\author{
%
Yung-Yu Chen
%
}

\begin{document}

\maketitle

\begin{abstract}
%
Summarize the schemes of the space-time conservation element and solution
element (CESE) method.
%
\end{abstract}

\section{Conservation Laws}
%
\label{s:conlaw}

The space-time conservation element and solution element (CESE) method solves
the conservation laws\cite{lax_hyperbolic_1973}.

\clearpage

\addcontentsline{toc}{chapter}{Bibliography}
%\bibliographystyle{myunsrtnat} % no sort (order in appearance)
\bibliographystyle{myplainnat} % sort by author
\bibliography{turgon_main}

\end{document}

% vim: set ai et sw=2 ts=2 tw=79:
