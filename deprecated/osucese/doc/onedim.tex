\documentclass[letterpaper,12pt,dvips]{article}
\usepackage[textwidth=6.5in,textheight=9in]{geometry}
\usepackage[colorlinks=true]{hyperref}
\usepackage{amsmath}
\usepackage{amssymb}
\usepackage{amsthm}
\usepackage{color}
\usepackage{vector}
\usepackage{graphicx}     % From LaTeX distribution
%\usepackage{subfigure}    % From CTAN/macros/latex/contrib/supported/subfigure
\usepackage{pst-all}      % From PSTricks
\usepackage{pst-poly}     % From pstricks/contrib/pst-poly
\usepackage{multido}      % From PSTricks
\usepackage[center,footnotesize]{caption}

\graphicspath{{../result_simple/}}

\numberwithin{equation}{section}

\begin{document}

\title{One Dimensional CESE Method}
\author{Yung-Yu Chen}
\date{2008.6.4}

\maketitle

\tableofcontents
%\listoffigures

\hspace{.5cm}

CESE is a well-developed algorithm to solve conservation laws.
The basic concept behind CESE is the conservation within space-time domain, 
which is fundamentally different from other traditional 
algorithms such as FVM, and also differs from non-conserving method such 
as FDM and FEM\cite{b:chang95}.
While a conservation law $u_t+[f(u)]_x=0$ can be rewritten as 
\begin{align*}
  \nabla\cdot(f(u),u) = 0
\end{align*}
for 2D space-time Euclidean space, in which $x_1=x,x_2=t$, by divergence 
theorem, the original conservation law over a space-time domain $V$ can 
be casted into flux conserving form over the closed boundary surface $S(V)$ 
surrounding $V$, as 
\begin{align*}
  \int_V\nabla\cdot\vec{h}dv = \oint_{S(V)}\vec{h}\cdot d\hat{\sigma} = 0, 
\end{align*}
where $\vec{h}\equiv(f(u),u)$.

Various codes has been developed using CESE\cite{b:chang95}.
Code accompanied by this document is yet another implementation of CESE, 
which solves for 1D Euler equations.
In order to well explain the algorithm in the implementation, this 
document is made as a supplement of the code, to exhibit the equations 
behind the lines.
Schemes described in this document includes $a$-$\mu$ scheme, $a$ scheme, 
$a$-$\varepsilon$ scheme\cite{b:chang95}, $c$ scheme, and 
$c$-$\tau$/$c$-$\tau^*$ scheme\cite{b:chang03}.
Corresponding schemes for weighted averaging, such as $\alpha$ 
scheme\cite{b:chang95}, Scheme-I, Scheme-II\cite{b:chang02}, W-1 scheme, 
and W-2 scheme\cite{b:chang03} are also stated.

This document is written as an introduction of CESE method, from the most 
basic form to recently developed schemes, and is organized as follows:
In section \ref{s:cese_intro}, the concept of CESE method is 
demonstrated with a single viscous model equation.
CESE for inviscid equations is described in section 
\ref{s:inviscid_model}.
Formulation of CESE for system of equations is constructed in section
\ref{s:system_eqn}, and Euler equations are used as the model equation.
Weighting functions for spatial derivatives are discussed in section 
\ref{s:weighted_average}.
The CFL-insensitive scheme of CESE is discussed in section \ref{s:ctau}, 
along with various weighting function.
In section \ref{s:nuni}, non-uniform spatial mesh formulation is 
introduced.
Local time-stepping is discussed in section \ref{s:lts}.
Finally, in section \ref{s:code}, the code itself is explained.

%%%%%%%%%%%%%%%%%%%%%%%%%%%%%%%%%%%%%%%%%%%%%%%%%%%%%%%%%%%%%%%%%%%%%%%%%%
%% Section 1
\section{Introduction to CESE method}
\label{s:cese_intro}
%%%% * Explain spatial decomposition: CE and SE.
%%%% * Use convection-diffusion equation as model equation to form 
%%%%   a-epsilon scheme, as demonstration of CESE method.
%%%%%%%%%%%%%%%%%%%%%%%%%%%%%%%%%%%%%%%%%%%%%%%%%%%%%%%%%%%%%%%%%%%%%%%%%%

Consider a 1D convection-diffusion equation, which is a simple case, 
\begin{align}
    \frac{\partial u}{\partial t} + a\frac{\partial u}{\partial x} 
  - \mu\frac{\partial^2u}{\partial x^2} = 0, \label{e:conv_diff_govern}
\end{align}
where the convection velocity $a$ and the viscosity coefficient $\mu$ 
($\ge0$) are constant.
With divergence theorem in space-time domain, the equation can be 
transformed to 
\begin{align}
  \oint_{S(V)}\vec{h}\cdot d\hat{\sigma}=0, \label{e:conservation}
\end{align}
where $\vec{h}\equiv(au-\mu\partial u/\partial x,u)$ and 
$d\hat{\sigma}=(dx, dt)$.

CESE method discretizes the integral equation in space-time domain by 
defining Conservation Elements (CEs).
In a single $\mathrm{CE}(j,n)$ (CE at $(x_j,t^n)$), the conservation of 
$\vec{h}$ can be approximated/written as 
\begin{align}
  \oint_{S(\mathrm{CE})}\vec{h}^*\cdot d\hat{\sigma} = 0, 
    \label{e:conserv_of_approx_h}
\end{align}
where $\vec{h}^*$ denotes the approximation function for $\vec{h}$.
Such a CE was regarded as Compounded Conservation Element (CCE, as 
dipicted in \figurename~\ref{f:cce}), and 
could be divided into two adjascent Basic Conservation Elements (BCEs) 
$\mathrm{CE}_-$ and $\mathrm{CE}_+$.
In BCEs, Equation (\ref{e:conserv_of_approx_h}) should also hold.
In another word, 
\begin{align*}
  \oint_{S(\mathrm{CE}_-)}\vec{h}^*\cdot d\hat{\sigma} = 0; \quad
  \oint_{S(\mathrm{CE}_+)}\vec{h}^*\cdot d\hat{\sigma} = 0, 
\end{align*}
where $S(\mathrm{CE}_{\pm})$ is the boundary surface surrounding 
$\mathrm{CE}_{\pm}$.

\begin{figure}[htbp]
\centering
\psset{unit=2cm}
\begin{pspicture}(-3,-.5)(3,1.5)
  \psset{linewidth=1pt}
  \psset{linecolor=black}
  \psframe(-1,0)(1,1)
  \psline[linestyle=dashed](0,0)(0,1)
  \psframe[linestyle=dotted,linecolor=red](-0.95,0.05)(-0.05,0.95)
  \rput[tr](-0.1,0.9){$\mathrm{CE}_-$}
  \psframe[linestyle=dotted,linecolor=red](0.05,0.05)(0.95,0.95)
  \rput[tr](0.9,0.9){$\mathrm{CE}_+$}
  \psframe[linestyle=dotted,linecolor=red](-1.05,-0.05)(1.05,1.05)
  \rput[tr](-1.1,0.9){$\mathrm{CE}$}
  \psdots[dotstyle=*](0,1)(-1,1)(-1,0)(0,0)(1,0)(1,1)(0,1)
  \uput{0.1}[u](0,1){A $(x_j,t^n)$}
  \uput{0.1}[ul](-1,1){B}
  \uput{0.1}[dl](-1,0){$(x_{j-\frac{1}{2}},t^{n-\frac{1}{2}})$ C}
  \uput{0.1}[d](0,0){D}
  \uput{0.1}[dr](1,0){E $(x_{j+\frac{1}{2}},t^{n-\frac{1}{2}})$}
  \uput{0.1}[ur](1,1){F}
  \psline[linestyle=dashed](0,0)(0,1)
\end{pspicture}
\caption{A compounded conservation element (CCE), containing two BCEs.}
\label{f:cce}
\end{figure}

Solution Elements (SEs) in CESE method were designed to provide a way to 
consistently evaluate approximated value of variables and functions.
In current setting, the shape of SEs is a cross mark around a mesh point 
$(j,n)$, as depicted in \figurename~\ref{f:crossse}.
The value of $u(x,t;j,n)$ within $\mathrm{SE}(j,n)$ (SE at space-time 
mesh $(j,n)$) would be approximated by 
\begin{align*}
  u^*(x,t;j,n) = u_j^n + (u_x)_j^n(x-x_j) + (u_t)_j^n(t-t^n), 
\end{align*}
where $u_j^n, (u_x)_j^n, (u_t)_j^n$ hold constant in $\mathrm{SE}(j,n)$, 
and $(x_j,t^n)$ are the coordinates of the mesh point $(j,n)$.
It should be noted that each CE is surrounded by SEs, so the evaluation of 
fluxes through boundary of CE depends only on the approximation of 
variables within SEs.

\begin{figure}[hbtp]
\centering
\psset{unit=1.5cm}
\begin{pspicture}(-1,-.5)(5,3.5)
  \psset{linewidth=1pt}
  \psset{linecolor=black}
  \psgrid[subgriddiv=1,gridlabels=0](0,0)(4,3)
  \uput[d](0,0){$j-1$}
  \uput[d](1,0){$j-\frac{1}{2}$}
  \uput[d](2,0){$j$}
  \uput[d](3,0){$j+\frac{1}{2}$}
  \uput[d](4,0){$j+1$}
  \uput[r](4,0){$n-1$}
  \uput[r](4,1){$n-\frac{1}{2}$}
  \uput[r](4,2){$n$}
  \uput[r](4,3){$n+\frac{1}{2}$}
  \def\PstSE{{%
    \psset{linewidth=1pt,linestyle=dotted,dotsep=1pt}
    \psdot(0,0)
    \psline[linestyle=dotted](.05,.05)(.05,.9)
    \psarc[linestyle=dotted](0,.9){0.05}{0}{180}
    \psline[linestyle=dotted](-.05,.9)(-.05,.05)
    \psline[linestyle=dotted](-.05,.05)(-.9,.05)
    \psarc[linestyle=dotted](-.9,0){0.05}{90}{270}
    \psline[linestyle=dotted](-.9,-.05)(-.05,-.05)
    \psline[linestyle=dotted](-.05,-.05)(-.05,-.9)
    \psarc[linestyle=dotted](0,-.9){.05}{180}{0}
    \psline[linestyle=dotted](.05,-.9)(.05,-.05)
    \psline[linestyle=dotted](.05,-.05)(.9,-.05)
    \psarc[linestyle=dotted](.9,0){.05}{270}{90}
    \psline[linestyle=dotted](.9,.05)(.05,.05)}}
  \rput(1,1){\PstSE}
  \rput(3,1){\PstSE}
  \rput(2,2){\PstSE}
\end{pspicture}
\caption{Cross-shaped solution element.}
\label{f:crossse}
\end{figure}

\subsection{Formulation for convection-diffusion equation}

Back to the convection diffusion eqation (Equation 
(\ref{e:conv_diff_govern})), 
the approximation for $u$ was required to fulfill the original 
differential equation, and since
\begin{align*}
  \frac{\partial u^*(x,t;j,n)}{\partial x} = (u_x)_j^n, \quad
  \frac{\partial u^*(x,t;j,n)}{\partial t} = (u_t)_j^n,
\end{align*}
for $(u_x)_j^n, (u_t)_j^n$ are both constant (such that $u^*_{xx}=0$), the 
relation between $(u_x)_j^n$ and $(u_t)_j^n$ can be obtained as 
\begin{align*}
              (u_t)_j^n + a(u_x)_j^n = 0
  \Rightarrow (u_t)_j^n = -a(u_x)_j^n\,.
\end{align*}
Further, the approximated $u^*$ can be rewritten as 
\begin{align*}
  u^*(x,t;j,n) &= u_j^n + (u_x)_j^n[(x-x_j) - a(t-t^n)]
\end{align*}
for $(x,t)\in\mathrm{SE}(j,n)$.
Consequently, $\vec{h}=(au-\mu\partial u/\partial x, u)$ can be 
approximated by
\begin{align*}
  \vec{h}^*(x,t;j,n) = 
  \left(
    au^*(x,t;j,n)-\mu\frac{\partial u^*(x,t;j,n)}{\partial x}, u^*(x,t;j,n)
  \right)\,.
\end{align*}

With $u^*$ and $\vec{h}^*$ on hand, the time-marching formula can be 
easily obtained by applying 
$\oint_{S(\mathrm{CE}_{\pm})}\vec{h}^*\cdot d\hat{\sigma}=0$ to each CCE, 
where $\vec{h}^*$ is 2 dimensional in this derivation, 
$S(\mathrm{CE}_{\pm}(j,n))$ is a closed curve in $E_2$ space.
By choosing to integrate counterclockwisely (flux flows outward CE), the 
flux vector can be evaluated to 
\begin{align*}
  d\hat{\sigma} 
  = \left[\begin{array}{cc} \cos90^{\circ} & \sin90^{\circ} \\ 
                           -\sin90^{\circ} & \cos90^{\circ} 
    \end{array}\right] d\hat{s}
  = \left[\begin{array}{cc} 0 & 1 \\ 
                           -1 & 0 
    \end{array}\right] 
    \left[\begin{array}{c} dx \\ dt 
    \end{array}\right]
  = (dt, -dx) \,.
\end{align*}
To simplify notation, define flux function
\begin{align*}
  F_{\pm} \equiv 
    \oint_{S(\mathrm{CE}_{\pm})}\vec{h}^*\cdot d\hat{\sigma} \,.
\end{align*}
Referring to \figurename~\ref{f:cce}, by evaluating 
$\vec{h}^*\cdot d\hat{\sigma}$ on each of the four sides 
$\overline{AB}, \overline{BC}, \overline{CD}, \overline{DA}$ of 
$\mathrm{CE}_-(j,n)$, the conservation function $F_-$ becomes
\begin{alignat*}{2}
  F_-(j,n) &= \frac{\Delta x^2}{4}\Big\{
    &&\frac{2}{\Delta x}\left(\frac{a\Delta t}{\Delta x}+1\right)
      \left(u_j^n-u_{j-1/2}^{n-1/2}\right) \\
  & &+\;& \frac{1}{2}\left[
          \frac{a^2\Delta t^2}{\Delta x^2} 
        - \frac{2\mu\Delta t} {\Delta x^2} - 1 
        \right](u_x)_j^n
    +     \frac{1}{2}\left[
          \frac{a^2\Delta t^2}{\Delta x^2} 
        + \frac{2\mu\Delta t} {\Delta x^2} - 1 
        \right](u_x)_{j-1/2}^{n-1/2}
    \Big\}\,.
\end{alignat*}
Simplify it by 
\begin{align}
  \nu &\equiv \frac{a\Delta t}{\Delta x}, \label{e:nu} \\
  \xi &\equiv \frac{4\mu\Delta t}{\Delta x^2}, \label{e:xi}
\end{align}
such that 
\begin{align*}
  \frac{4}{\Delta x^2}F_-(j,n) &= 
    -\frac{1}{2}\left[
      (1-\nu^2+\xi)(u_x)_j^n + (1-\nu^2-\xi)(u_x)_{j-1/2}^{n-1/2}
    \right] \\
  &\quad\quad 
  + \frac{2(\nu+1)}{\Delta x}\left(u_j^n-u_{j-1/2}^{n-1/2}\right) \,. 
\end{align*}
$F_+$ for $\mathrm{CE}_+$ can be evaluated in the same way.
They can be written together as
\begin{align}
  \frac{2}{\Delta x}F_{\pm}(j,n) &= 
    \pm\left[
        (1-\nu^2+\xi)\frac{\Delta x}{4}(u_x)_j^n
      + (1-\nu^2-\xi)\frac{\Delta x}{4}(u_x)_{j\pm1/2}^{n-1/2}
    \right] \notag \\
  &\quad\quad
  + (1\mp\nu)\left(u_j^n-u_{j\pm1/2}^{n-1/2}\right)\,. 
    \label{e:both_conserv_func}
\end{align}
There are two equations for two unknowns $u_j^n, (u_x)_j^n$.
A formula for the unknowns can be obtained by solving this system of 
equations.
Let
\begin{alignat*}{2}
  \alpha_1 &= 1-\nu^2+\xi,\quad &\alpha_2 &= 1-\nu^2-\xi, \\
  \beta_1 &= 1-\nu, &\beta_2 &= 1+\nu, \\
  q_1 &= x_j^n, &q_2 &= \frac{\Delta x}{4}(u_x)_j^n, \\
  a_1 &= x_{j-1/2}^{n-1/2}, 
 &a_2 &= \frac{\Delta x}{4}(x_n)_{j-1/2}^{n-1/2}, \\
  b_1 &= x_{j+1/2}^{n-1/2}, 
 &b_2 &= \frac{\Delta x}{4}(x_n)_{j+1/2}^{n-1/2}, 
\end{alignat*}
and Equation (\ref{e:both_conserv_func}) becomes
\begin{align*}
  \beta_2q_1 - \alpha_1q_2 &= \beta_2a_1 + \alpha_2a_2, \\
  \beta_1q_1 + \alpha_1q_2 &= \beta_1b_1 - \alpha_2b_2\,.
\end{align*}
The solution can be rewritten in vector form
\begin{align*}
  \vec{q}(j,n) = 
    Q_+\vec{q}(j-\frac{1}{2},n-\frac{1}{2})
  + Q_-\vec{q}(j+\frac{1}{2},n-\frac{1}{2}), 
\end{align*}
where
\begin{align*}
  Q_+ &\equiv 
    \frac{1}{2}
    \left[\begin{array}{cc}
      1+\nu                                     & 
      1-\nu^2-\xi                               \\
      \frac{-(1-\nu^2)}{1-\nu^2+\xi}            & 
      \frac{-(1-\nu)(1-\nu^2-\xi)}{1-\nu^2+\xi}
    \end{array}\right], \\
  Q_- &\equiv 
    \frac{1}{2}
    \left[\begin{array}{cc}
      1-\nu                                     & 
      -(1-\nu^2-\xi)                            \\
      \frac{(1-\nu^2)}{1-\nu^2+\xi}             & 
      \frac{-(1+\nu)(1-\nu^2-\xi)}{1-\nu^2+\xi}
    \end{array}\right], \\
  \vec{q}(j,n) &\equiv 
    \left[\begin{array}{c}
      u_j^n \\ \frac{\Delta x}{4}(u_x)_j^n 
    \end{array}\right]\,.
\end{align*}

By expanding the vector form of solution, the time-marching formula can be 
written as
\begin{alignat}{2}
  u_j^n &= &\frac{1}{2}\Big[
    &(1+\nu)u_{j-\frac{1}{2}}^{n-\frac{1}{2}} 
    + (1-\nu^2-\xi)(u_{\bar{x}})_{j-\frac{1}{2}}^{n-\frac{1}{2}} 
    \notag \\
  &&+ &(1-\nu)u_{j+\frac{1}{2}}^{n-\frac{1}{2}}
    - (1-\nu^2-\xi)(u_{\bar{x}})_{j+\frac{1}{2}}^{n-\frac{1}{2}}
           \Big] \label{e:amuu},
\end{alignat}
\begin{alignat}{2}
  (u_{\bar{x}})_j^n &= &\frac{1}{2(1-\nu^2+\xi)}\Big[
    -& (1-\nu^2)u_{j-\frac{1}{2}}^{n-\frac{1}{2}}
    -  (1-\nu)(1-\nu^2-\xi)(u_{\bar{x}})_{j-\frac{1}{2}}^{n-\frac{1}{2}} 
    \notag \\
  &&+& (1-\nu^2)u_{j+\frac{1}{2}}^{n-\frac{1}{2}}
    -  (1+\nu)(1-\nu^2-\xi)(u_{\bar{x}})_{j+\frac{1}{2}}^{n-\frac{1}{2}}
                       \Big] \label{e:amuux}, 
\end{alignat}
where $u_{\bar{x}}\equiv\frac{\Delta x}{4}u_x$.
Remind that $\nu\equiv\frac{a\Delta t}{\Delta x}$, 
$\xi\equiv\frac{4\nu\Delta t}{\Delta x^2}$.
This set of equations is referred as $a$-$\mu$ scheme in 
\cite{b:chang95}.

%%%%%%%%%%%%%%%%%%%%%%%%%%%%%%%%%%%%%%%%%%%%%%%%%%%%%%%%%%%%%%%%%%%%%%%%%%
%% Section 2.
\section{Inviscid model equation}
\label{s:inviscid_model}
%%%% * Introduce time-reversible a scheme with inviscid model equation.
%%%% * Show how to add numerical dissipation to inviscid scheme to 
%%%%   achieve a time-irreversible a-epsilon scheme, along with the 
%%%%   formulation.
%%%%%%%%%%%%%%%%%%%%%%%%%%%%%%%%%%%%%%%%%%%%%%%%%%%%%%%%%%%%%%%%%%%%%%%%%%

If the original model equation (Equation (\ref{e:conv_diff_govern})) 
is changed to 
\begin{align}
  \frac{\partial u}{\partial t} + a\frac{\partial u}{\partial x} = 0, 
    \label{e:conv_govern}
\end{align}
which is the inviscid convection equation, a time-reversable formulation 
can be obtained by set $\xi=0$ in Equation (\ref{e:amuu}), 
(\ref{e:amuux}), as 
\begin{align}
  u_j^n &= \frac{1}{2}\Big[
      (1+\nu)u_{j-\frac{1}{2}}^{n-\frac{1}{2}} 
    + \frac{\Delta x}{4}(1-\nu^2)(u_x)_{j-\frac{1}{2}}^{n-\frac{1}{2}}
    + (1-\nu)u_{j+\frac{1}{2}}^{n-\frac{1}{2}}
    - \frac{\Delta x}{4}(1-\nu^2)(u_x)_{j+\frac{1}{2}}^{n-\frac{1}{2}}
  \Big], \label{e:au}
\end{align}
\begin{align}
  (u_x)_j^n = \frac{1}{\Delta x/2}\Big\{
      \Big[ u_{j+\frac{1}{2}}^{n-\frac{1}{2}}
         - (1+\nu)(u_{\bar{x}})_{j+\frac{1}{2}}^{n-\frac{1}{2}} \Big]
    - \Big[ u_{j-\frac{1}{2}}^{n-\frac{1}{2}}
         + (1-\nu)(u_{\bar{x}})_{j-\frac{1}{2}}^{n-\frac{1}{2}} \Big]
  \Big\} \label{e:aux}\,.
\end{align}
These two time-marching formulae are called $a$ scheme.

Since $a$ scheme is time-reversible, it is not suitable to be used to 
simulate problems that is not reversible in time, such as shock.
To overcome this problem, artificial viscosity is added into $a$ scheme 
in a controllable fashion.
The conservation relation within a CCE will be adjusted to 
\begin{align}
  F_{\pm}(j,n) = \pm\frac{\varepsilon(1-\nu^2)\Delta x^2}{4}(du_x)_j^n, 
    \label{e:aeconserv}
\end{align}
where
\begin{align*}
  (du_x)_j^n \equiv 
    \frac{(u_x)_{j+1/2}^{n-1/2} + (u_x)_{j-1/2}^{n-1/2}}{2}
  - \frac{u_{j+1/2}^{n-1/2} - u_{j-1/2}^{n-1/2}}        {\Delta x}, 
\end{align*}
such that 
\begin{align*}
    \oint_{S(\mathrm{CE}  )}\vec{h}^*\cdot d\hat{\sigma} &= 0, \\
    \oint_{S(\mathrm{CE}_-)}\vec{h}^*\cdot d\hat{\sigma} &= 
  - \oint_{S(\mathrm{CE}_+)}\vec{h}^*\cdot d\hat{\sigma} \,.
\end{align*}
This modified scheme is referred as $a$-$\varepsilon$ in \cite{b:chang95}.

\subsection{Formulation for artificial viscosity}

In order to build the time-marching formula for $a$-$\varepsilon$ scheme,
recall Equation (\ref{e:both_conserv_func}) for the expression of 
$F_{\pm}$, and let $\mu=0\Rightarrow\xi=0$, such that
\begin{align*}
  \frac{4}{\Delta x^2}F_{\pm}(j,n) 
    = \pm\frac{(1-\nu^2)[(u_x)_j^n+(u_x)_{j\pm1/2}^{n-1/2}]}{2}
    + \frac{2(1\mp\nu)}{\Delta x}(u_j^n-u_{j\pm1/2}^{n-1/2})\,.
\end{align*}
Use Equation (\ref{e:aeconserv}) to form two relationships within a CCE 
\begin{align*}
  &(1-\nu^2)\frac{\Delta x}{4}[(u_x)_j^n+(u_x)_{j+1/2}^{n-1/2}]
    + (1-\nu)(u_j^n-u_{j+1/2}^{n-1/2}) \\
  &\quad = \varepsilon(1-\nu^2)\frac{\Delta x}{4}
           [(u_x)_{j+1/2}^{n-1/2}+(u_x)_{j-1/2}^{n-1/2}]
         - \frac{1}{2}
           \varepsilon(1-\nu^2)(u_{j+1/2}^{n-1/2}-u_{j-1/2}^{n-1/2}), \\
  &-(1-\nu^2)\frac{\Delta x}{4}[(u_x)_j^n+(u_x)_{j-1/2}^{n-1/2}]
    + (1+\nu)(u_j^n-u_{j-1/2}^{n-1/2}) \\
  &\quad = -\varepsilon(1-\nu^2)\frac{\Delta x}{4}
            [(u_x)_{j+1/2}^{n-1/2}+(u_x)_{j-1/2}^{n-1/2}]
          + \frac{1}{2}
            \varepsilon(1-\nu^2)(u_{j+1/2}^{n-1/2}-u_{j-1/2}^{n-1/2})\,.
\end{align*}
Let 
\begin{alignat*}{2}
  &       q_1 = u_j^n, 
  &\quad& q_2 = \frac{\Delta x}{4}(u_x)_j^n, \\
  &       a_1 = x_{j-1/2}^{n-1/2}, 
  &&      a_2 = \frac{\Delta x}{4}(u_x)_{j-1/2}^{n-1/2}, \\
  &       b_1 = x_{j+1/2}^{n-1/2}, 
  &&      b_2 = \frac{\Delta x}{4}(u_x)_{j+1/2}^{n-1/2}, \\
  &       \alpha = 1-\nu^2, && \\
  &       \beta_1 = 1-\nu, 
  &&      \beta_2 = 1+\nu,
\end{alignat*}
the equations can be rewritten as
\begin{align*}
  \alpha(q_2+b_2) + \beta_1(q_1-b_1) 
    &=   \varepsilon\alpha(b_2+a_2) 
       - \frac{1}{2}\varepsilon\alpha(b_1-a_1), \\
  -\alpha(q_2+a_2) + \beta_2(q_1-a_1) 
    &=  -\varepsilon\alpha(b_2+a_2) 
       + \frac{1}{2}\varepsilon\alpha(b_1-a_1), 
\end{align*}
and the unknows $q_1, q_2$ are solved to
\begin{align*}
  \left[\begin{array}{c} q_1 \\ q_2 \end{array}\right]
    = \frac{1}{2}\left[\begin{array}{cc} 1+\nu & 1-\nu^2 \\
      \varepsilon-1 & 2\varepsilon-1+\nu \end{array}\right]
      \left[\begin{array}{c} a_1 \\ a_2 \end{array}\right]
    + \frac{1}{2}\left[\begin{array}{cc} 1-\nu & 1-\nu^2 \\
      1-\varepsilon & 2\varepsilon - 1 - \nu \end{array}\right]
  \left[\begin{array}{c} b_1 \\ b_2 \end{array}\right]\,.
\end{align*}

By expanding the vector form of solution, the time-marching formula for 
$a$-$\varepsilon$ scheme is obtained as 
\begin{align}
  u_j^n &= \frac{1}{2}\Big[
      (1+\nu)u_{j-\frac{1}{2}}^{n-\frac{1}{2}} 
    + \frac{\Delta x}{4}(1-\nu^2)(u_x)_{j-\frac{1}{2}}^{n-\frac{1}{2}}
    + (1-\nu)u_{j+\frac{1}{2}}^{n-\frac{1}{2}}
    - \frac{\Delta x}{4}(1-\nu^2)(u_x)_{j+\frac{1}{2}}^{n-\frac{1}{2}}
  \Big], \label{e:aeu}
\end{align}
\begin{align}
  (u_x)_j^n = \frac{(u')_{j+1/2}^n - (u')_{j-1/2}^n}{\Delta x}
            + (2\varepsilon-1)(du_x)_j^n, \label{e:aeux}
\end{align}
where
\begin{align*}
  (u')_{j\pm1/2}^n \equiv
    u_{j\pm1/2}^{n-1/2}+\frac{\Delta t}{2}(u_t)_{j\pm1/2}^{n-1/2}\,.
\end{align*}
Note that in $a$-$\varepsilon$ scheme, $u_j^n$ term is evaluated in the 
same way as in $a$ scheme.
The difference between $a$-$\varepsilon$ scheme and $a$ scheme lies on 
the evaluation of $(u_x)_j^n$ term.

\subsection{Simplified scheme}

By letting $\varepsilon=0.5$, the $(du_x)_j^n$ term in the formala 
evaluating $(u_x)_j^n$ (Equation (\ref{e:aeux})) vanishes.
With this special $\varepsilon=0.5$ specified, a new scheme named $c$ 
scheme is obtained by incorporating the following formulae
\begin{align}
  u_j^n &= \frac{1}{2}\Big[
      (1+\nu)u_{j-\frac{1}{2}}^{n-\frac{1}{2}} 
    + \frac{\Delta x}{4}(1-\nu^2)(u_x)_{j-\frac{1}{2}}^{n-\frac{1}{2}}
    + (1-\nu)u_{j+\frac{1}{2}}^{n-\frac{1}{2}}
    - \frac{\Delta x}{4}(1-\nu^2)(u_x)_{j+\frac{1}{2}}^{n-\frac{1}{2}}
  \Big], \label{e:cu}
\end{align}
\begin{align}
  (u_x)_j^n = \frac{(u')_{j+1/2}^n - (u')_{j-1/2}^n}{\Delta x}
  \,. \label{e:cux}
\end{align}

%%%%%%%%%%%%%%%%%%%%%%%%%%%%%%%%%%%%%%%%%%%%%%%%%%%%%%%%%%%%%%%%%%%%%%%%%%
%% Section 3
\section{Formulation for system of equations}
\label{s:system_eqn}
%%%% * Formulate for vector form of coupling equations with a-epsilon 
%%%%   scheme and c scheme.
%%%% * Jacobian and eigenvalues for 1D Euler equations.
%%%%%%%%%%%%%%%%%%%%%%%%%%%%%%%%%%%%%%%%%%%%%%%%%%%%%%%%%%%%%%%%%%%%%%%%%%

Up to here, the model equations (Equation (\ref{e:conv_diff_govern}), 
(\ref{e:conv_govern})) being dealt with are both scaler equations.
In order to apply CESE method to various conservation laws in physics 
and engineering, the formulation is going to be extended to a system 
of equations.

Consider the general form of the system of conservation laws for 1D case, 
\begin{align}
  \frac{\partial u_m}{\partial t} + \frac{\partial f_m}{\partial x} = 0
    , \label{e:syswave}
\end{align}
where $m=1,\ldots,n$ and $f_m(u_1,\ldots,u_n)$.
By defining $\vec{h}_m\equiv(f_m,u_m)$, the differential system equations
can be rewritten in the space-time integral form for control volume $V$
\begin{align*}
\oint_{S(V)}\vec{h}_m\cdot d\hat{\sigma} = 0\,.
\end{align*}
In a SE, $u_m, f_m$ are approximated as 
\begin{align*}
  & u_m^*(x,t;j,n) \equiv (u_m)_j^n + (u_{mx})_j^n(x-x_j)
                        + (u_{mt})_j^n(t-t^n), \\
  & f_m^*(x,t;j,n) \equiv (f_m)_j^n + (f_{mx})_j^n(x-x_j)
                        + (f_{mt})_j^n(t-t^n)  \,.
\end{align*}
Referring to the CCE $\square BCEF$ in \figurename~\ref{f:cce} that 
contains two BCEs $\mathrm{CE}_-$, $\mathrm{CE}_+$, two relationships 
can be formed as
\begin{align*}
  \oint_{S(\mathrm{CE}_-)} \vec{h}_m^*\cdot d\hat{\sigma} &= 
    -\frac{\varepsilon(1-\nu^2)\Delta x^2}{4}(du_{mx})_j^n, \\
  \oint_{S(\mathrm{CE}_+)} \vec{h}_m^*\cdot d\hat{\sigma} &= 
    \frac{\varepsilon(1-\nu^2)\Delta x^2}{4}(du_{mx})_j^n, 
\end{align*}
where
\begin{align}
  (du_{mx})_j^n 
    \equiv \frac{(u_{mx})_{j+1/2}^{n-1/2} 
                + (u_{mx})_{j-1/2}^{n-1/2}}{2}
  - \frac{(u_m)_{j+\frac{1}{2}}^{n-\frac{1}{2}}
        - (u_m)_{j-\frac{1}{2}}^{n-\frac{1}{2}}}
         {\Delta x}, \label{e:dumx}
\end{align}
and Courant number $\nu$ should be determined by the maximum in the system 
of equations.
By evaluating the sum of the two equations
\begin{align*}
    \oint_{S(\mathrm{CE}_-)} \vec{h}^*\cdot d\hat{\sigma} 
  + \oint_{S(\mathrm{CE}_+)} \vec{h}^*\cdot d\hat{\sigma} = 0, 
\end{align*}
the formulation for $u_j^n$ is obtained as 
\begin{alignat*}{2}
  (u_m)_j^n &= \frac{1}{2}\Big[
      &&(u_m)_{j-1/2}^{n-1/2}+(u_m)_{j+1/2}^{n-1/2} \\
  & &+ &\left( 
        \frac{\Delta t}{\Delta x}(f_m)_{j-1/2}^{n-1/2} 
      + \frac{\Delta t^2}{4\Delta x}(f_{mt})_{j-1/2}^{n-1/2}
      + \frac{\Delta x}{4}(u_{mx})_{j-1/2}^{n-1/2} 
      \right) \\
  & &- &\left( 
        \frac{\Delta t}{\Delta x}(f_m)_{j+1/2}^{n-1/2} 
      + \frac{\Delta t^2}{4\Delta x}(f_{mt})_{j+1/2}^{n-1/2}
      + \frac{\Delta x}{4}(u_{mx})_{j+1/2}^{n-1/2} 
      \right) 
    \Big]\,.
\end{alignat*}
To simplify the formulation, let
\begin{align}
  (s_m)_j^n \equiv \frac{\Delta t}{\Delta x}(f_m)_j^n 
    + \frac{\Delta t^2}{4\Delta x}(f_{mt})_j^n
    + \frac{\Delta x}{4}(u_{mx})_j^n, \label{e:sm}
\end{align}
such that
\begin{align*}
  (u_m)_j^n = \frac{1}{2}\left[
      (u_m)_{j-1/2}^{n-1/2} + (u_m)_{j+1/2}^{n-1/2}
    + (s_m)_{j-1/2}^{n-1/2} - (s_m)_{j+1/2}^{n-1/2} 
    \right]\,.
\end{align*}

In order to evaluate $u_{mt}, f_{mx}, f_{mt}$, recall that 
$u_m^*, f_m^*$ follow $u_{mt}+f_{mx}=0$ (Equation (\ref{e:syswave})), such 
that
\begin{align}
  (u_{mt})_j^n = -(f_{mx})_j^n\,. \label{e:sysumt}
\end{align}
By chain rule
\begin{align}
  f_{mx} = f_{m,k}u_{kx}, \label{e:sysfmk}
\end{align}
where 
\begin{align}
  f_{m,k} \equiv \frac{\partial f_m}{\partial u_k} \label{e:sysjacobian}
\end{align}
is the Jacobian.
Similarly, $f_{mt}$ can be written as
\begin{align}
  f_{mt} = f_{m,k}u_{kt} = -f_{m,k}f_{k,l}u_{lx}\,. \label{e:sysfmt}
\end{align}

As to the spatial derivative, directly use the generalization of the 
scaler analog of $(u_{mx})_j^n$ to have
\begin{align*}
  (u_{mx})_j^n = \frac{(u_m')_{j+1/2}^n-(u_m')_{j-1/2}^n}{\Delta x}
    + (2\varepsilon-1)(du_{mx})_j^n, 
\end{align*}
where
\begin{align}
  (u_m')_{j\pm1/2}^n &\equiv 
      (u_m)_{j\pm1/2}^{n-1/2} 
    + \frac{\Delta t}{2}(u_{mt})_{j\pm1/2}^{n-1/2}\,. \label{e:ump}
\end{align}

In summary, the formulation of 
\begin{align}
  (u_m)_j^n &= \frac{1}{2}\left[
      (u_m)_{j-1/2}^{n-1/2} + (u_m)_{j+1/2}^{n-1/2}
    + (s_m)_{j-1/2}^{n-1/2} - (s_m)_{j+1/2}^{n-1/2} 
    \right], \label{e:sysaeu} \\
  (u_{mx})_j^n &= \frac{(u_m')_{j+1/2}^n-(u_m')_{j-1/2}^n}{\Delta x}
    + (2\varepsilon-1)(du_{mx})_j^n, \label{e:sysaeux}
\end{align}
where
\begin{align*}
  (s_m)_j^n &= 
      \frac{\Delta t}  {\Delta x} (f_m)_j^n 
    + \frac{\Delta t^2}{4\Delta x}(f_{mt})_j^n
    + \frac{\Delta x}  {4}        (u_{mx})_j^n, \\
  (du_{mx})_j^n &=
      \frac{(u_{mx})_{j+1/2}^{n-1/2} 
                  + (u_{mx})_{j-1/2}^{n-1/2}}{2}
    - \frac{(u_m)_{j+\frac{1}{2}}^{n-\frac{1}{2}}
          - (u_m)_{j-\frac{1}{2}}^{n-\frac{1}{2}}}
           {\Delta x}, \\
  (u_m')_{j\pm1/2}^n &=
      (u_m)_{j\pm1/2}^{n-1/2} 
    + \frac{\Delta t}{2}(u_{mt})_{j\pm1/2}^{n-1/2}, 
\end{align*}
forms $a$-$\varepsilon$ scheme for Equation 
(\ref{e:syswave})\cite{b:chang95}.
Also, by letting $\varepsilon=0.5$, $c$ scheme can be also obtained from 
these two equations.

For system of equations, the CFL number $\nu$ is determined to be the 
eigenvalue of the Jacobian (Equation (\ref{e:sysjacobian})) of the 
maximal absolute value.

\subsection{1D Euler equations}

The Euler equations are
\begin{alignat}{3}
  & \frac{\partial}{\partial t}(\rho)
    &\,+\,& \frac{\partial}{\partial x}(\rho v) &\;=\;& 0
    , \label{e:euler1} \\
  & \frac{\partial}{\partial t}(\rho v)
    &\,+\,& \frac{\partial}{\partial x}(p+\rho v^2) &\;=\;& 0
    , \label{e:euler2} \\
  & \frac{\partial}{\partial t}(\frac{p}{\gamma-1}+\frac{\rho v^2}{2})
    &\,+\,& \frac{\partial}{\partial x}
      (\frac{\gamma}{\gamma-1}pv+\frac{1}{2}\rho v^3) &\;=\;& 0
    \,. \label{e:euler3}
  \end{alignat}
Let
\begin{align*}
  u_1 \equiv \rho, 
  u_2 \equiv \rho v, 
  u_3 \equiv \frac{p}{\gamma-1} + \frac{1}{2}\rho v^2,
\end{align*}
and
\begin{align*}
  &f_1 \equiv u_2, \\
  &f_2 \equiv (\gamma-1)u_3 + \frac{3-\gamma}{2}\frac{u_2^2}{u_1}, \\
  &f_3 \equiv \gamma\frac{u_2u_3}{u_1} 
             - \frac{\gamma-1}{2}\frac{u_2^3}{u_1^2}, 
\end{align*}
and then the Euler equations can be expressed as a system of conservation 
laws (Equation (\ref{e:syswave})).
Define 
\begin{align}
  F \equiv 
    \left[\begin{array}{ccc}
      f_{1,1} & f_{1,2} & f_{1,3} \\
      f_{2,1} & f_{2,2} & f_{2,3} \\
      f_{3,1} & f_{3,2} & f_{3,3} 
    \end{array}\right]
    = 
    \left[\begin{array}{ccc}
      0 & 1 & 0 \\
      \frac{\gamma-3}{2}\frac{u_2^2}{u_1^2} & 
      -(\gamma-3)\frac{u_2}{u_1} & 
      \gamma-1 \\
      (\gamma-1)\frac{u_2^3}{u_1^3} - \gamma\frac{u_2u_3}{u_1^2} & 
      \gamma\frac{u_3}{u_1} - \frac{3}{2}(\gamma-1)\frac{u_2^2}{u_1^2} & 
      \gamma\frac{u_2}{u_1}
    \end{array}\right], \label{e:euler_jacobian}
\end{align}
as the Jacobian matrix.
The eigenvalues of $F$ can be calculated from the characeristic 
polynomial
\begin{align*}
  |F-\lambda| = 
    \left|\begin{array}{ccc}
      -\lambda & 1 & 0 \\
      \frac{\gamma-3}{2}\frac{u_2^2}{u_1^2} 
      & -(\gamma-3)\frac{u_2}{u_1} - \lambda
      & \gamma-1 \\
      (\gamma-1)\frac{u_2^3}{u_1^3} - \gamma\frac{u_2u_3}{u_1^2}
      & \gamma\frac{u_3}{u_1} - \frac{3}{2}(\gamma-1)\frac{u_2^2}{u_1^2}
      & \gamma\frac{u_2}{u_1} - \lambda
    \end{array}\right| = 0\,.
\end{align*}
Since 
\begin{align*}
  & \frac{u_2}{u_1} = v, \\
  & \frac{u_3}{u_1} = \frac{1}{\gamma-1}\frac{p}{\rho} + \frac{1}{2}v^2
    = \frac{1}{\gamma(\gamma-1)}c^2 + \frac{1}{2}v^2, 
\end{align*}
then
\begin{align*}
  |F-\lambda| &= 
    \left|\begin{array}{ccc}
      -\lambda & 1 & 0 \\
      \frac{\gamma-3}{2}v^2 & 
      -(\gamma-3)v - \lambda & 
      \gamma-1 \\
      (\gamma-1)v^3 
        - \gamma v(\frac{c^2}{\gamma(\gamma-1)}+\frac{v^2}{2}) &
      \gamma(\frac{c^2}{\gamma(\gamma-1)}+\frac{v^2}{2})
        - \frac{3}{2}(\gamma-1)v^2 &
      \gamma v - \lambda
    \end{array}\right| \\
  &= -\lambda^3 + 3v\lambda^2 + (c^2-3v^2)\lambda + v(v^2-c^2) 
   = 0, 
\end{align*}
such that
\begin{align}
  \lambda = 
  \left\{\begin{aligned} 
    &v   , \\
    &v+c , \\
    &v-c .
  \end{aligned}\right. \label{e:euler_eigenvalue}
\end{align}

%%%%%%%%%%%%%%%%%%%%%%%%%%%%%%%%%%%%%%%%%%%%%%%%%%%%%%%%%%%%%%%%%%%%%%%%%%
%% Section 4
\section{Weighted average}
\label{s:weighted_average}
%%%% * The origin of weighting functions for evaluation of spatial 
%%%%   derivative.
%%%%%%%%%%%%%%%%%%%%%%%%%%%%%%%%%%%%%%%%%%%%%%%%%%%%%%%%%%%%%%%%%%%%%%%%%%

In $a$ scheme and $c$ scheme, the formulae (Equation (\ref{e:aux}),
(\ref{e:cux})) for spatial derivative $(u_x)_j^n$ can be regarded as 
simply central-differencing analogy\cite{b:chang95, b:chang03}.
With discontinuity, numerical wiggles may occur in the 
solution\cite{b:chang03}.
The wiggles can be supressed by introducing weighting 
functions\cite{b:chang95, b:chang03}.

Take $c$ scheme for example. 
The spatial derivative is computed with Equation (\ref{e:cux})
\begin{align*}
  (u_x)_j^n = \frac{(u')_{j+1/2}^n - (u')_{j-1/2}^n}{\Delta x}\,.
\end{align*}
If define
\begin{align}
  (u_{x-})_j^n &\equiv 
    \frac{u_j^n - (u')_{j-\frac{1}{2}}^n}{\Delta x/2}, \label{e:ux-} \\
  (u_{x+})_j^n &\equiv 
    \frac{(u')_{j+\frac{1}{2}}^n - u_j^n}{\Delta x/2}, \label{e:ux+}
\end{align}
then the spatial derivative can be rewritten as 
\begin{align*}
  (u_x)_j^n = \frac{1}{2}(u_{x-})_j^n + \frac{1}{2}(u_{x+})_j^n, 
\end{align*}
which is a special case for the weighted average in the form of 
\begin{align}
  (u_x)_j^n = (w_-)_j^n(u_{x-})_j^n + (w_+)_j^n(u_{x+})_j^n
    , \label{e:weighting}
\end{align}
where the weight factors $(w_-)_j^n=(w_+)_j^n=\frac{1}{2}$.
By designing weight factors, $(u_x)_j^n$ can be evaluated to be closer 
to the value with smaller absolute value of $(u_{x-})_j^n$ and 
$(u_{x+})_j^n$.
With the designed weight factors, the absolute value of $(u_x)_j^n$ is 
smaller than that calculated with simple average.
The local numerical dissipation introduced in this way annihilates 
the numerical wiggle and overshoot\cite{b:chang03}.

\subsection{alpha/Scheme-I/W-1}

In the literatures, various weighting functions were 
designed\cite{b:chang95, b:chang02, b:chang03}.
The simplest one is 
\begin{align}
  (w_-)_j^n = 
    \frac{|(u_{x+})_j^n|^{\alpha}}
         {|(u_{x-})_j^n|^{\alpha}+|(u_{x+})_j^n|^{\alpha}}, \quad
  (w_+)_j^n = 
    \frac{|(u_{x-})_j^n|^{\alpha}}
         {|(u_{x-})_j^n|^{\alpha}+|(u_{x+})_j^n|^{\alpha}}, \label{e:alpha}
\end{align}
where $\alpha\ge0$.
It is the original $\alpha$ scheme referred as Equation (4.39) in 
\cite{b:chang95}.
Also, Scheme-I (Equation (2.21) in \cite{b:chang02}) and W-1 scheme 
(Equation (2.38a), (2.38b) in \cite{b:chang03}) are basically $\alpha$
scheme, but they are specific to the $c$-$\tau$ scheme.

\subsection{Scheme-II}

Scheme-I/W-1 scheme is flawed since it can not supress wiggle/overshoot 
while $|\nu|\rightarrow0$.
Another weighting scheme named Scheme-II was proposed to remedy it.
The formulation for Scheme-II is
\begin{align}
  w_- = \frac{[1+f(|\nu|)(s_+)_j^n]}
             {2 + f(|\nu|)[(s_-)_j^n+(s_+)_j^n]}, \quad
  w_+ = \frac{[1+f(|\nu|)(s_-)_j^n]}
             {2 + f(|\nu|)[(s_-)_j^n+(s_+)_j^n]}, \label{e:Scheme-II}
\end{align}
where 
\begin{align}
  (s_{\pm})_j^n &\equiv 
    \frac{|(u_{\bar{x}\pm})_j^n|}
         {\min\left(|(u_{\bar{x}-})_j^n|, 
                    |(u_{\bar{x}+})_j^n|\right)} - 1 
    \ge 0, \label{e:Scheme-IIs} \\
  f(|\nu|) &\equiv 
    \frac{1}{2|\nu|}\,. \label{e:Scheme-IIf}
\end{align}
Remind that $(u_{\bar{x}})_j^n\equiv \frac{\Delta x}{4}(u_x)_j^n$.
Remarks:
\begin{itemize}
\item While calculating $f(|\nu|)$, add a tiny value in the denominator, 
to prevent error of divided by zero.
\item Also, while calculating $(s_{\pm})_j^n$, add a tiny value both in 
numerator and denominator, to prevent evaluating 
$\lim_{x\rightarrow0}\frac{x}{x}=1$ wrongly to 0.
\end{itemize}
(See also Equation (3.23), (3.26), (3.27) in \cite{b:chang02}.)

\subsection{W-2}

W-2 scheme is different to W-1 scheme only for CFL insensitive scheme 
$c$-$\tau$ scheme in the following Section~\ref{s:ctau}.
W-2 scheme still uses Equation (\ref{e:alpha})
\begin{align*}
  (w_-)_j^n = 
    \frac{|(u_{x+})_j^n|^{\alpha}}
         {|(u_{x-})_j^n|^{\alpha}+|(u_{x+})_j^n|^{\alpha}}, \quad
  (w_+)_j^n = 
    \frac{|(u_{x-})_j^n|^{\alpha}}
         {|(u_{x-})_j^n|^{\alpha}+|(u_{x+})_j^n|^{\alpha}}\,.
\end{align*}
However, instead of using the value to be weighted (calculated from 
$(u')_j^n(P_-)$ and $(u')_j^n(P_+)$) to calculate the weighting factors, 
the spatial difference that is not moved to $P_{\pm}$ (calculated from 
unmoved $(u')_{j\pm\frac{1}{2}}^n$) is used for weighting factor 
calculation.
(See also section 4.2 in \cite{b:chang03}.)

\subsection{W-3}

Derived from W-1, W-3 scheme tries to amplify weighting factor with the
deviation of itself.
Define
\begin{align}
  \delta_{\pm} \equiv w_{\pm} - \frac{1}{2} \label{e:W3_delta}
\end{align}
with $w_- + w_+ = 1$, such that $\delta_- + \delta_+ = 0$.
Let $\delta = |\delta_{\pm}|$, the maximum amplification factor can be 
defined as $\sigma_{\text{max}} \equiv \frac{1}{2\delta}$.
The real amplification factor can be then chosen to 
\begin{align}
  \sigma = \min\left\{\frac{1}{2\delta}, \frac{\sigma_0}{|\nu|}\right\}, 
    \label{e:W3_sigma}
\end{align}
where $\sigma_0\sim1$.
The amplified weighting factors can be determined by
\begin{align}
  w_{\pm}' = \frac{1}{2} + \sigma\delta_{\pm}\,. \label{e:W3_w}
\end{align}
(See also Equation (4.16), (4.30), (4.32), (4.25) in \cite{b:chang03}.)

\subsection{W-4}

Scheme W-4 also tries to amplify the weighting factors, but in different 
way to W-3.
For $w_- = w_1 = \frac{s_1}{s_1+s_2}$, $w_+ = w_2 = \frac{s_2}{s_1+s_2}$, 
$s_1<s_2$, define
\begin{align}
  \eta \equiv \frac{s_2}{s_1} - 1\,. \label{e:W4_eta}
\end{align}
With the amplification factor 
\begin{align}
  \sigma = \frac{\sigma_0}{|\nu|}, \label{e:W4_sigma}
\end{align}
where $\sigma_0 \sim 1$ (as W-3), the weighting factors can be amplified 
as
\begin{align}
  &\tilde{s}_1 = s_1, \tilde{s}_2 = (1+\sigma\eta)\tilde{s}_1, 
    \label{e:W4_s} \\
  \Rightarrow\quad &\tilde{w}_l = \frac{\tilde{s}_l}{\sum_i\tilde{s}_i}
    \label{e:W4_w} \,.
\end{align}
While coding this scheme up, some branching statements (if's) are 
needed.
(See also Equation (4.35), (4.37), (4.38), (4.39), (4.46) in 
\cite{b:chang03}.)

\subsection{Comparison of W-series weighting schemes}

Some comments of the 4 W-series weighting schemes can be made from the 
simulation result ($\nu\approx0.0004$).
\begin{itemize}
  \item Significant overshoot can be confirmed\cite{b:chang03} in W-1 
        scheme (see \figurename~\ref{f:W-1}).
  \item W-3 scheme is OK, but the numerical diffusion is larger than 
        W-2 and W-4 around the discontinuity.
  \item Despite the potential problem in W-2 (see the last two paragraph 
        in the section 4.2 of \cite{b:chang03}), its performance is the 
        best, even better than W-4.
\end{itemize}

\begin{figure}[htbp]
\centering
\includegraphics[width=0.8\textwidth]{simple_NuniW1_4.0e-6_1.0.eps}
\caption{W-1 scheme.}
\label{f:W-1}
\includegraphics[width=0.8\textwidth]{simple_NuniW2_4.0e-6_1.0.eps}
\caption{W-2 scheme.}
\label{f:W-2}
\end{figure}

\begin{figure}[htbp]
\centering
\includegraphics[width=0.8\textwidth]{simple_NuniW3_4.0e-6_1.0.eps}
\caption{W-3 scheme.}
\label{f:W-3}
\includegraphics[width=0.8\textwidth]{simple_NuniW4_4.0e-6_1.0.eps}
\caption{W-4 scheme.}
\label{f:W-4}
\end{figure}

%%%%%%%%%%%%%%%%%%%%%%%%%%%%%%%%%%%%%%%%%%%%%%%%%%%%%%%%%%%%%%%%%%%%%%%%%%
%% Section 5
\section{CFL insensitive schemes}
\label{s:ctau}
%%%% * Introduce CFL insensitive c-tau scheme.
%%%%%%%%%%%%%%%%%%%%%%%%%%%%%%%%%%%%%%%%%%%%%%%%%%%%%%%%%%%%%%%%%%%%%%%%%%

The original $a$-$\varepsilon$ scheme and $c$ scheme have a problem of 
severe smearing for small Courant (CFL) number ($\nu$).
An ``ideal solver'' was proposed in \cite{b:chang03} named $c$-$\tau$ 
scheme to remedy this problem.

$c$-$\tau$ scheme is simply a scheme that evaluates $(u_x)_j^n$ with 
$c$ scheme (Equation (\ref{e:cux})) while $|\nu|\rightarrow1$, and with 
$a$ scheme (Equation (\ref{e:aux})) while $\nu\rightarrow0$.
Note that $|\nu|<1$.
With $\nu\rightarrow0$, the formula for spatial derivative 
(Equation (\ref{e:aux})) becomes
\begin{align*}
  (u_x)_j^n = \frac{1}{\Delta x/2}\Big\{
      \Big[ u_{j+\frac{1}{2}}^{n-\frac{1}{2}}
         - \frac{\Delta x}{4}(u_x)_{j+\frac{1}{2}}^{n-\frac{1}{2}} 
      \Big]
    - \Big[ u_{j-\frac{1}{2}}^{n-\frac{1}{2}}
         + \frac{\Delta x}{4}(u_x)_{j-\frac{1}{2}}^{n-\frac{1}{2}} 
      \Big]
  \Big\}, 
\end{align*}
and can be furthur written as
\begin{align*}
  (u_x)_j^n = 
    \frac{\left[(u')_j^n(M_+) - (u')_j^n(M_-)\right]}{\Delta x/2}
\end{align*}
since $(u')_j^n(M_{\pm}) \rightarrow u_j^{n-\frac{1}{2}}(M_{\pm})$ as 
$\nu\rightarrow0$, where $M_{\pm}$ is the spatial mid-point of 
$\mathrm{CE}_{\pm}$, as depicted in \figurename~\ref{f:ctau}.

\begin{figure}[htbp]
\centering
\psset{unit=3cm}
\begin{pspicture}(-2.5,-.5)(2.5,1.5)
  \psset{linewidth=1pt}
  \psset{linecolor=black}
  \footnotesize
  \psframe(-2,0)(2,1)
  \psline[linestyle=dashed](0,0)(0,1)
  \psdots[dotstyle=*](0,1)(-1,1)(-1.4,1)(-2,1)(-2,0)(0,0)(2,0)(2,1)(1.4,1)(1,1)(0,1)
  \uput{0.1}[u](0,1){A}
  \uput{0.1}[u](0,1.2){$(j,n)$}
  \uput{0.1}[u](-1,1){$\mathrm{M}_-$}
  \uput{0.1}[u](-1.4,1){$\mathrm{P}_-$}
  \uput{0.1}[u](-2,1){B}
  \uput{0.1}[u](-2,1.2){$(j-\frac{1}{2},n)$}
  \uput{0.1}[d](-2,0){C}
  \uput{0.1}[d](-2,-.2){$(j-\frac{1}{2},n-\frac{1}{2})$}
  \uput{0.1}[d](0,0){D}
  \uput{0.1}[d](2,0){E}
  \uput{0.1}[d](2,-.2){$(j+\frac{1}{2},n-\frac{1}{2})$}
  \uput{0.1}[u](2,1){F}
  \uput{0.1}[u](2,1.2){$(j+\frac{1}{2},n)$}
  \uput{0.1}[u](1.4,1){$\mathrm{P}_+$}
  \uput{0.1}[u](1,1){$\mathrm{M}_+$}
  \psline[linestyle=dashed](0,0)(0,1)

  \scriptsize
  \psline[linewidth=.5pt,linestyle=dotted,dotsep=1pt](-1,1)(-1,.3)
  \psline[linewidth=.5pt,linestyle=dotted,dotsep=1pt](-1.4,1)(-1.4,.3)
  \psline[linewidth=.5pt,linestyle=dotted,dotsep=1pt](1,1)(1,.3)
  \psline[linewidth=.5pt,linestyle=dotted,dotsep=1pt](1.4,1)(1.4,.3)
  % Left marks.
  \psline[linewidth=.5pt]{<->}(-2,.2)(0,.2)
  \rput(-1,.2){\psframebox*{$\frac{\Delta x}{2}$}}
  \psline[linewidth=.5pt]{<->}(-2,.4)(-1,.4)
  \rput(-1.5,.4){\psframebox*{$\frac{\Delta x}{4}$}}
  \psline[linewidth=.5pt]{<->}(-1,.4)(0,.4)
  \rput(-.5,.4){\psframebox*{$\frac{\Delta x}{4}$}}
  \psline[linewidth=.5pt]{<->}(-1.4,.6)(-2,.6)
  \rput(-1.7,.75){\psframebox*{$(1-\tau)\frac{\Delta x}{4}$}}
  \psline[linewidth=.5pt]{<->}(-1.4,.6)(0,.6)
  \rput(-.8,.75){\psframebox*{$(1+\tau)\frac{\Delta x}{4}$}}
  % Right marks.
  \psline[linewidth=.5pt]{<->}(2,.2)(0,.2)
  \rput(1,.2){\psframebox*{$\frac{\Delta x}{2}$}}
  \psline[linewidth=.5pt]{<->}(2,.4)(1,.4)
  \rput(1.5,.4){\psframebox*{$\frac{\Delta x}{4}$}}
  \psline[linewidth=.5pt]{<->}(1,.4)(0,.4)
  \rput(.5,.4){\psframebox*{$\frac{\Delta x}{4}$}}
  \psline[linewidth=.5pt]{<->}(1.4,.6)(2,.6)
  \rput(1.7,.75){\psframebox*{$(1-\tau)\frac{\Delta x}{4}$}}
  \psline[linewidth=.5pt]{<->}(1.4,.6)(0,.6)
  \rput(.8,.75){\psframebox*{$(1+\tau)\frac{\Delta x}{4}$}}
\end{pspicture}
\caption{Definition of points for CFL insensitive scheme.}
\label{f:ctau}
\end{figure}

In $c$-$\tau$ scheme, the spatial derivative is evaluated with 
\begin{align}
  (u_x)_j^n = 
    \frac{(u')_j^n(P_+) - (u')_j^n(P_-)}
         {(1+\tau)\Delta x/2}, \label{e:ctauux}
\end{align}
where
\begin{align}
  (u')_j^n(P_{\pm}) \equiv 
        (u')_{j\pm\frac{1}{2}}^n 
    \mp (1-\tau)\frac{\Delta x}{4}(u_x)_{j\pm\frac{1}{2}}^{n-\frac{1}{2}}
  \,. \label{e:upp}
\end{align}
Also referring to \figurename~\ref{f:ctau}, point $P_{\pm}$ is floated 
with the parameter $0\le\tau\le1$.
As $\tau\rightarrow0$, $P_{\pm}\rightarrow M_{\pm}$, the value evaluated 
by Equation (\ref{e:ctauux}) is approaching to $a$ scheme; on the other 
hand, as $\tau\rightarrow1$, $P_{\pm}\rightarrow(j\pm\frac{1}{2},n)$, the 
value evaluated is approaching to $c$ scheme.
By designing the strictly increasing function $\tau = \tau(s)$ 
carefully\cite{b:chang03}, the ``ideal solver'' requirement is achieved.

Generally, there are infinitely many choices for the parameter function 
$\tau=\tau(s)$ in $c$-$\tau$ scheme.
Specifically, 
\begin{align}
  \tau = \tau(|\nu|) = |\nu| \label{e:ctau*}
\end{align}
can be one of such functions, and 
is referred as $c$-$\tau^*$ scheme in \cite{b:chang02, b:chang03}.

In summary, the time-marching formulation for $c$-$\tau^*$ scheme are 
Equation (\ref{e:cu}), (\ref{e:ctauux}), along with the definition 
Equation (\ref{e:upp}), (\ref{e:ctau*}), and can be written as
\begin{align*}
  u_j^n = \frac{1}{2}\Big[
      (1+\nu)u_{j-\frac{1}{2}}^{n-\frac{1}{2}} 
    + \frac{\Delta x}{4}(1-\nu^2)(u_x)_{j-\frac{1}{2}}^{n-\frac{1}{2}}
    + (1-\nu)u_{j+\frac{1}{2}}^{n-\frac{1}{2}}
    - \frac{\Delta x}{4}(1-\nu^2)(u_x)_{j+\frac{1}{2}}^{n-\frac{1}{2}}
  \Big], 
\end{align*}
\begin{align*}
  (u_x)_j^n &= 
    \frac{\left[(u')_j^n(P_+) - (u')_j^n(P_-)\right]}
         {(1+\tau)\Delta x/2}, \\
  (u')_j^n(P_{\pm}) &\equiv 
        (u')_{j\pm\frac{1}{2}}^n 
    \mp (1-\tau)\frac{\Delta x}{4}
        (u_x)_{j\pm\frac{1}{2}}^{n-\frac{1}{2}}, \\
  \tau &\equiv |\nu|\,.
\end{align*}

\subsection{Modification to weighting function}

Various schemes to calculated the weighted average of spatial derivative 
can be deployed to $c$-$\tau$/$c$-$\tau^*$ scheme.
The simplest one is the $\alpha$ scheme as described with 
Equation (\ref{e:alpha}).
From $\alpha$ scheme, Scheme-I\cite{b:chang02} and W-1 
scheme\cite{b:chang03} were developed to use Equation (\ref{e:alpha}) 
\begin{align*}
  (w_{\pm})_j^n = 
    \frac{|({u_x}_{\mp})_j^n|^{\alpha}}
         {|({u_x}_-)_j^n|^{\alpha}+|({u_x}_+)_j^n|^{\alpha}}, 
\end{align*}
with Equation (\ref{e:ux-}), (\ref{e:ux+}) replaced with 
\begin{align}
  (u_{x-})_j^n &\equiv 
    \frac{u_j^n - (u')_j^n(P_-)}{(1+\tau)\Delta x/4}, \label{e:uxp-} \\
  (u_{x+})_j^n &\equiv 
    \frac{(u')_j^n(P_+) - u_j^n}{(1+\tau)\Delta x/4}\,. \label{e:uxp+}
\end{align}
While applying other weighting schemes (except W-2 scheme), just use the 
new $({u_{x}}_{\pm})_j^n$.

\subsection{Extension to system of equations}

While system of equations as Equation (\ref{e:syswave}) is considered, 
the formulation for $c$-$\tau^*$ scheme is similar to 
$a$-$\varepsilon$/$c$ scheme.
Equation (\ref{e:sysaeu}) along with Equation (\ref{e:sm}) can be 
directly used as the formula for $(u_m)_j^n$, 
\begin{align*}
  (u_m)_j^n &= \frac{1}{2}\left[
      (u_m)_{j-1/2}^{n-1/2} + (u_m)_{j+1/2}^{n-1/2}
    + (s_m)_{j-1/2}^{n-1/2} - (s_m)_{j+1/2}^{n-1/2} 
    \right], \\
  (s_m)_j^n &= 
      \frac{\Delta t}  {\Delta x} (f_m)_j^n 
    + \frac{\Delta t^2}{4\Delta x}(f_{mt})_j^n
    + \frac{\Delta x}  {4}        (u_{mx})_j^n\,.
\end{align*}
$(u_{mx})_j^n$ is time-marched with the analogous formulation to 
Equation (\ref{e:ctauux}) along with Equation (\ref{e:upp}), as
\begin{align}
  (u_{mx})_j^n &= 
    \frac{(u_m')_j^n(P_+)-(u_m')_j^n(P_-)}
         {(1+\tau)\Delta x/2}, \\
  (u_m')_j^n(P_{\pm}) &= 
        (u_m')_{j\pm\frac{1}{2}}^n 
    \mp (1-\tau)\frac{\Delta x}{4}(u_x)_{j\pm\frac{1}{2}}^{n-\frac{1}{2}}, 
\end{align}
and $\tau=|\nu|$.
In order to time-march $(u_{mx})_j^n$, the CFL number ($\nu$) in 
$c$-$\tau^*$ scheme has to be determined for the system of equations.
Take the 1D Euler equations (Equation (\ref{e:euler1}), (\ref{e:euler2}), 
(\ref{e:euler3})) for example, 
\begin{align}
\nu \equiv \left\{
  \begin{aligned}
    &(v_j^n+c_j^n)\frac{\Delta t}{\Delta x} 
      \quad\text{for}\quad v_j^n \ge 0, \\
    &(v_j^n-c_j^n)\frac{\Delta t}{\Delta x} 
      \quad\text{for}\quad v_j^n < 0, \\
  \end{aligned}
  \right. \label{e:eulernu}
\end{align}
where
\begin{align}
  v_j^n &= 
    \frac{(u_2)_j^n}{(u_1)_j^n}, \label{e:eulerv} \\
  c_j^n &= \sqrt{\frac{\gamma p}{\rho}} = 
    \sqrt{\gamma(\gamma-1)
          \left[
              \frac{(u_3)_j^n}{(u_1)_j^n} 
            - \frac{1}{2}\left(\frac{(u_2)_j^n}{(u_1)_j^n}\right)^2
          \right]}\,. \label{e:eulerc}
\end{align}

\section{Non-uniform spatial mesh}
\label{s:nuni}

CESE has been extended to non-uniform spatial mesh\cite{b:chang03}.
Generally, in the non-uniform mesh, the CCE can be arbitrarily 
non-symmetric, as depicted in \figurename~\ref{f:nonuni}.
The variables are evaluated on the ``solution points'', which is marked 
by $\times$ in \figurename~\ref{f:nonuni}.
This document has a convention to use superscript $s$ to denote value 
on solution points.
For example, $x_j^s$ denotes for the spatial location of the solution 
point of $j$-th CCE.
It should be noted that in the two BCEs within the single CCE, the mid 
mesh points are defined as
\begin{align*}
  x_j^{\pm} = \frac{x_{j\pm\frac{1}{2}} + x_j}{2}\,.
\end{align*}

\begin{figure}[htbp]
\centering
\psset{unit=2.7cm}
\begin{pspicture}(-3,-.5)(3,1.5)
  \psset{linewidth=1pt}
  \psset{linecolor=black}
  \scriptsize

  % CCE frame.
  \psframe(-2,0)(2,1)
  \psline[linestyle=dashed](-.5,0)(-.5,1)
  \psdots[dotstyle=*](-.5,1)
  \uput{4pt}[u](-.5,1){$(j,n)$}
  \psdots[dotstyle=*](-2,0)
  \uput{4pt}[dr](-2,0){$(j-\frac{1}{2},n-\frac{1}{2})$}
  \psdots[dotstyle=*](2,0)
  \uput{4pt}[dl](2,0){$(j+\frac{1}{2},n-\frac{1}{2})$}
  % Solution point of CE(j,n).
  \psdots[dotstyle=x](0,1)
  \uput{4pt}[u](0,1){$(j,n)^s$}
  % Half position of CE_- and CE_+.
  \psdots[dotstyle=o](-1.25,1)
  \uput{4pt}[u](-1.25,1){$(j,n)^-$}
  \psdots[dotstyle=o](.75,1)
  \uput{4pt}[u](.75,1){$(j,n)^+$}
  % Solution points of CE(j\pm\frac{1}{2},n-\frac{1}{2}).
  \psline[linestyle=dashed,linewidth=.5pt](-2,0)(-2.5,0)
  \psdots[dotstyle=x](-2.4,0)
  \uput{4pt}[d](-2.4,0){$(j-\frac{1}{2},n-\frac{1}{2})^s$}
  \psline[linestyle=dashed,linewidth=.5pt](2,0)(2.5,0)
  \psdots[dotstyle=x](2.4,0)
  \uput{4pt}[d](2.4,0){$(j+\frac{1}{2},n-\frac{1}{2})^s$}
  % P^{\pm}.
  \psdots[dotstyle=square](-1.8,1)
  \uput{4pt}[u](-1.8,1){$P_j^{n-}$}
  \psdots[dotstyle=square](1.5,1)
  \uput{4pt}[u](1.5,1){$P_j^{n+}$}

  % Rulers.
  \psline[linewidth=.5pt,linestyle=dotted,dotsep=1pt](-2,1)(-2,1.3)
  \psline[linewidth=.5pt,linestyle=dotted,dotsep=1pt](2,1)(2,1.3)
  \psline[linewidth=.5pt,linestyle=dotted,dotsep=1pt](-1.25,1)(-1.25,.4)
  \psline[linewidth=.5pt,linestyle=dotted,dotsep=1pt](-1.8,1)(-1.8,.7)
  \psline[linewidth=.5pt,linestyle=dotted,dotsep=1pt](.75,1)(.75,.4)
  \psline[linewidth=.5pt,linestyle=dotted,dotsep=1pt](1.5,1)(1.5,.7)
  \psline[linewidth=.5pt,linestyle=dotted,dotsep=1pt](-2.4,0)(-2.4,1)
  \psline[linewidth=.5pt,linestyle=dotted,dotsep=1pt](2.4,0)(2.4,1)
  % \Delta x_j
  \psline[linewidth=.5pt]{<->}(-2,1.25)(2,1.25)
  \rput(0,1.25){\psframebox*{$\Delta x_j$}}
  % \Delta x_j^-
  \psline[linewidth=.5pt]{<->}(-2,.2)(-.5,.2)
  \rput(-1.25,.2){\psframebox*{$\Delta x_j^-$}}
  % \Delta x_j^+
  \psline[linewidth=.5pt]{<->}(-.5,.2)(2,.2)
  \rput(.75,.2){\psframebox*{$\Delta x_j^+$}}
  % \tau(x_j^--x_{j-\frac{1}{2})^s}
  \psline[linewidth=.5pt]{<->}(-2.4,.5)(-1.25,.5)
  \rput(-1.8,.5){\psframebox*{$(x_j^- - x_{j-\frac{1}{2}}^s)$}}
  % \tau(x_{j+\frac{1}{2})^s-x_j^+}
  \psline[linewidth=.5pt]{<->}(.75,.5)(2.4,.5)
  \rput(1.6,.5){\psframebox*{$(x_{j+\frac{1}{2}}^s - x_j^+)$}}
  % \tau(x_j^--x_{j-\frac{1}{2})^s}
  \psline[linewidth=.5pt]{<->}(-2.4,.85)(-1.8,.85)
  \uput{2pt}[r](-1.8,.8){\psframebox*{$(1-\tau)(x_j^- - x_{j-\frac{1}{2}}^s)$}}
  % \tau(x_{j+\frac{1}{2})^s-x_j^+}
  \psline[linewidth=.5pt]{<->}(1.5,.85)(2.4,.85)
  \uput{2pt}[l](1.5,.8){\psframebox*{$(1-\tau)(x_{j+\frac{1}{2}}^s - x_j^+)$}}
\end{pspicture}
\caption{Definition of points non-uniform scheme.}
\label{f:nonuni}
\end{figure}

A general formulation for 1D Euler equations is given in the section 6 of
\cite{b:chang08}.
The marching formulae for each half time step are 
\begin{alignat}{2}
u_m^s &= 
    &&\frac{\Delta x_j^-}{\Delta x_j}
      \left[
          (u_m^s)_{j-\frac{1}{2}}^{n-\frac{1}{2}}
        + (x_j^- - x_{j-\frac{1}{2}}^s)
          (u_{mx})_{j-\frac{1}{2}}^{n-\frac{1}{2}}
      \right] \notag \\
& &+ &\frac{\Delta x_j^+}{\Delta x_j}
      \left[
          (u_m^s)_{j+\frac{1}{2}}^{n-\frac{1}{2}}
        + (x_{j+\frac{1}{2}}^s - x_j^+)
          (u_{mx})_{j+\frac{1}{2}}^{n-\frac{1}{2}}
      \right] \notag \\
& &+ &\frac{\Delta t}{2\Delta x_j}(s_m)_{j-\frac{1}{2}}^{n-\frac{1}{2}}
   -  \frac{\Delta t}{2\Delta x_j}(s_m)_{j+\frac{1}{2}}^{n-\frac{1}{2}}
    \label{e:nonunisys_us}
\end{alignat}
where
\begin{align*}
  (s_m)_j^n \equiv (f_m)_j^n + \frac{\Delta t}{4}(f_{mt})_j^n \,.
\end{align*}
Note that $(f_m)_j$ is $f_m$ on $j$-th ``mesh'' point, so it should 
be evaluated from value at solution point as
\begin{align*}
  (f_m)_j^n = (f_m^s)_j^n + (x_j - x_j^s)(f_{mx})_j^n, 
\end{align*}
such that
\begin{align}
  (s_m)_j^n \equiv (f_m^s)_j^n + (x_j - x_j^s)(f_{mx})_j^n 
                 + \frac{\Delta t}{4}(f_{mt})_j^n \,.
    \label{e:nonunisys_s}
\end{align}
(See also Equation (6.39), (6.40) in \cite{b:chang08}.)
The advance of $u_{mx}$ is given by
\begin{align}
  (u_{mx})_j^n = \frac{u_m'({P_j^n}^+)-u_m'({P_j^n}^-)}
                      {x  ({P_j^n}^+)-x  ({P_j^n}^-)}, 
    \label{e:nonunisys_ux}
\end{align}
where
\begin{align}
  x({P_j^n}^{\pm}) &= 
      x_j^{\pm} + \tau(x_{j\pm\frac{1}{2}}^s-x_j^{\pm}), 
    \label{e:nonunisys_xp} \\
  u_m'({P_j^n}^{\pm}) &= 
      (u_m^s)_{j\pm\frac{1}{2}}^{n-\frac{1}{2}}
    + \frac{\Delta t}{2}(u_{mt})_{j\pm\frac{1}{2}}^{n-\frac{1}{2}}
    + \left[x({P_j^n}^{\pm})-x_{j\pm\frac{1}{2}}^s\right]
      (u_{mx})_{j\pm\frac{1}{2}}^{n-\frac{1}{2}} \,. 
    \label{e:nonunisys_upp}
\end{align}
(See also Equation (6.43), (6.33), (6.41), (6.42) in \cite{b:chang08}.)
For the calculation of weighted averaged (Equation \ref{e:weighting}), 
define
\begin{align}
  (u_{mx-})_j^n &\equiv
    \frac{(u_m^s)_j^n-u_m'({P_j^n}^-)}{x_j^s-x({P_j^n}^-)}, \\
  (u_{mx+})_j^n &\equiv
    \frac{u_m'({P_j^n}^+)-(u_m^s)_j^n}{x({P_j^n}^+)-x_j^s}\,.
\end{align}
(See also Equation (6.44), (6.45) in \cite{b:chang08}.)
CFL number should be determined by either 
\begin{align}
  [\nu^{(1)}]_j^n = 
    \max\left\{
      \frac{(c_j^n+v_j^n)\Delta t}{x_{j+1}^s-x_j^s}, 
      \frac{(c_j^n-v_j^n)\Delta t}{x_j^s-x_{j-1}^s}
    \right\}, \label{e:sysnonuni_nu1}
\end{align}
or
\begin{align}
  [\nu^{(2)}]_j^n = 
    (c_j^n+v_j^n)\max\left\{
      \frac{\Delta t}{x_{j+1}^s-x_j^s}, 
      \frac{\Delta t}{x_j^s-x_{j-1}^s}
    \right\}\,. \label{e:sysnonuni_nu2}
\end{align}
If on the boundary, $x_{j+1}^s-x_j^s$, $x_j^s-x_{j-1}^s$ in Equation 
(\ref{e:sysnonuni_nu1}), (\ref{e:sysnonuni_nu2}) should be replaced to 
$2(x_{j+\frac{1}{2}}^s-x_j^s)$, $2(x_j^s-x_{j-\frac{1}{2}}^s)$.
(See also Equation (6.62), (6.64) in \cite{b:chang08}.)

\section{Local time-stepping}
\label{s:lts}

\section{Sample code}
\label{s:code}

The code is written in Python\cite{b:Python} and FORTRAN 90/95 
programming languages mixed.
Python is a fully object-oriented and dynamic language, while 
FORTRAN 90/95 is basically a sequential language.
In the code, Python is responsible to architect of the software, and 
most of the data structure and input/output are done in the Python part.
On the other hand, algorithm for CESE method is implemented in FORTRAN.
The mixture of Python and FORTRAN unleashes the vast functionality 
offered by Python, and takes the advantage of the performance and clear 
representation of algorithm for stencil operation provided by FORTRAN.

\subsection{How to run}

The code right now depends on several software/packages.
Python version 2.5 should be installed in system, and working C and 
FORTRAN compilers should be available too, while gcc and gfortran are 
recommended and tested.
numpy, scipy and matplotlib have to be installed, and the newest version 
is recommended.

Python code does not need to be compiled, but FORTRAN code needs.
To compile the FORTRAN part of code, issue \verb+python2.5 build.py build+
in the root directory (of the extracted package).
Under Windows operating system, the command \verb+python2.5+ does not 
work, and \verb+python+ should be used instead.

After the code compiled, issue \verb+python2.5 simple.py+ to simulate the 
simple shock tube problem.
The script \verb+simple.py+ itself defines the initial conditions and 
all the logic to run the cases.
It should be taken as a setting file for a ``caseset'', but just a 
little longer.

\subsection{Code structure}

The code tries to separate CESE algorithm out of other logic as much as 
possible.
The core of the code is \verb+core1d+ Python packge, which is actually 
the directory of the same name in the root directory.
CESE algorithm is modeled in the \verb+core1d.cese+ module (in Python, 
a module is generally a file containing Python code), in which there is 
a hierarchy of classes that defines CESE solver.
Each solver incorporates a time-marching subroutine residing in 
\verb+core1d._core+ external module, which is written in FORTRAN and 
wrapped by numpy.f2py package for the use from Python code.

All the contents in \verb+core1d._core+ are defined in the files in 
\verb+core1d/_core_fsrc+ directory, and each file contains only a 
subroutine.
Those subroutines define all the schemes used in the current code.

\begin{thebibliography}{99}
\bibitem{b:chang95} Chang, S.-C. (1995), 
``The Method of Space-Time Conservation Element and Solution Element -- 
A New Approach for Solving the Navier-Stokes and Euler Equation.''
Journal of Computational Physics 119(2): 295-324.
\bibitem{b:chang02} Chang, S.-C. (2002, July),
``Courant number insensitive ce/se schemes.'' AIAA Paper 2002-3890.
\bibitem{b:chang03} Chang, S.-C. and Wang, X.-Y. (2003), 
``Multi-dimensional courant number insensitive ce/se euler solvers for 
applications involving highly nonuniform meshes,'' 
AIAA Paper 2003-5285.
\bibitem{b:chang05} Chang, S.-C., Wu, Y., Yang, V., and Wang, X.-Y., 
``Local time-stepping procedures for the space-time conservation element 
and solution element method,'' 
International Journal of Computational Fluid Dynamics, 
vol. 19, no. 5, pp. 359-380, July 2005.
\bibitem{b:chang08} Chang, S.-C., not yet published script.
\bibitem{b:Python} \url{http://www.python.org/}.
\end{thebibliography}

\end{document}
